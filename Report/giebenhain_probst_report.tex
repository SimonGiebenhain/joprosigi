% This is samplepaper.tex, a sample chapter demonstrating the
% LLNCS macro package for Springer Computer Science proceedings;
% Version 2.20 of 2017/10/04
%
\documentclass[runningheads]{llncs}
%
\usepackage{graphicx}
% Used for displaying a sample figure. If possible, figure files should
% be included in EPS format.
%
% If you use the hyperref package, please uncomment the following line
% to display URLs in blue roman font according to Springer's eBook style:
% \renewcommand\UrlFont{\color{blue}\rmfamily}

\begin{document}
%
\title{Advanced Data Challenge Seminar\\
		Avito Demand Prediction}
%

% If the paper title is too long for the running head, you can set
% an abbreviated paper title here
%
\author{Simon Giebenhain \and Jonas Probst}
%

% First names are abbreviated in the running head.
% If there are more than two authors, 'et al.' is used.
%
\institute{University of Konstanz\\
\email{simon.giebenhain@uni-konstanz.de}\\
\email{jonas.probst@uni-konstanz.de}}
%
\maketitle              % typeset the header of the contribution
%
\begin{abstract}
This report gives a step-by-step description on how we tackled the Avito Demand Prediction Challenge on Kaggle and gives an overview of the used models, one LightGBM and one Neural Network. The goal of the challenge was to predict 'deal probability' for an advertisement on Avito based on it's parameters. 

\keywords{Kaggle \and Avito \and Demand Prediction \and Data Challenge}
\end{abstract}
%
%
%
\section{Dataset and Data Schema}
\subsection{Dataset}
Kaggle provided the following files for the challenge:
\begin{itemize}
	\item \textbf{train.csv:} Contains approximately 1.5 million training examples.
	\item \textbf{test.csv:} Contains approximately 500 000 test examples.
	\item \textbf{train\_active and test\_active:} Supplemental Data from ads that were displayed in the same time periods.
	\item \textbf{periods\_train and periods\_test:} Supplemental data showing the dates when the ads from train\_active and test\_active were activated and displayed.
	\item \textbf{train\_jpg.zip and test\_jpg.zip:} Images from ads in train.csv and test.csv.
	\item \textbf{sample\_submission.csv:} A sample submission in the correct format.\\
	\begin{center}
	\begin{tabular}{|c|c|}
		\hline 
		item\_id & deal\_probability \\ 
		\hline 
		1 & 0 \\ 
		\hline 
		4 & 0.1 \\ 
		\hline 
		126 & 0.6 \\ 
		\hline 
		... & ... \\ 
		\hline 
	\end{tabular} 
	\end{center} 
\end{itemize}

\subsection{Data Schema}  

\paragraph{Sample Heading (Fourth Level)}
The contribution should contain no more than four levels of
headings. Table~\ref{tab1} gives a summary of all heading levels.

\begin{table}
\caption{Table captions should be placed above the
tables.}\label{tab1}
\begin{tabular}{|l|l|l|}
\hline
Heading level &  Example & Font size and style\\
\hline
Title (centered) &  {\Large\bfseries Lecture Notes} & 14 point, bold\\
1st-level heading &  {\large\bfseries 1 Introduction} & 12 point, bold\\
2nd-level heading & {\bfseries 2.1 Printing Area} & 10 point, bold\\
3rd-level heading & {\bfseries Run-in Heading in Bold.} Text follows & 10 point, bold\\
4th-level heading & {\itshape Lowest Level Heading.} Text follows & 10 point, italic\\
\hline
\end{tabular}
\end{table}


\noindent Displayed e quations are centered and set on a separate
line.
\begin{equation}
x + y = z
\end{equation}
Please try to avoid rasterized images for line-art diagrams and
schemas. Whenever possible, use vector graphics instead (see



\begin{theorem}
This is a sample theorem. The run-in heading is set in bold, while
the following text appears in italics. Definitions, lemmas,
propositions, and corollaries are styled the same way.
\end{theorem}
%
% the environments 'definition', 'lemma', 'proposition', 'corollary',
% 'remark', and 'example' are defined in the LLNCS documentclass as well.
%
\begin{proof}
Proofs, examples, and remarks have the initial word in italics,
while the following text appears in normal font.
\end{proof}
For citations of references, we prefer the use of square brackets
and consecutive numbers. Citations using labels or the author/year
convention are also acceptable. The following bibliography provides
a sample reference list with entries for journal
articles~\cite{ref_article1}, an LNCS chapter~\cite{ref_lncs1}, a
book~\cite{ref_book1}, proceedings without editors~\cite{ref_proc1},
and a homepage~\cite{ref_url1}. Multiple citations are grouped
\cite{ref_article1,ref_lncs1,ref_book1},
\cite{ref_article1,ref_book1,ref_proc1,ref_url1}.
%
% ---- Bibliography ----
%
% BibTeX users should specify bibliography style 'splncs04'.
% References will then be sorted and formatted in the correct style.
%
% \bibliographystyle{splncs04}
% \bibliography{mybibliography}
%
\begin{thebibliography}{8}
\bibitem{ref_article1}
Author, F.: Article title. Journal \textbf{2}(5), 99--110 (2016)

\bibitem{ref_lncs1}
Author, F., Author, S.: Title of a proceedings paper. In: Editor,
F., Editor, S. (eds.) CONFERENCE 2016, LNCS, vol. 9999, pp. 1--13.
Springer, Heidelberg (2016). \doi{10.10007/1234567890}

\bibitem{ref_book1}
Author, F., Author, S., Author, T.: Book title. 2nd edn. Publisher,
Location (1999)

\bibitem{ref_proc1}
Author, A.-B.: Contribution title. In: 9th International Proceedings
on Proceedings, pp. 1--2. Publisher, Location (2010)

\bibitem{ref_url1}
LNCS Homepage, \url{http://www.springer.com/lncs}. Last accessed 4
Oct 2017
\end{thebibliography}
\end{document}